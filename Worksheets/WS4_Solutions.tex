\documentclass[12pt]{article}

% Opening
\title{Applied Combinatorics Homework 4}
\author{Akash Narayanan}
\usepackage{amsmath, amsfonts, amssymb, amsthm, enumitem, tikz}
\usepackage{caption, subcaption, float}

% Problem environment
\theoremstyle{definition}
\newtheorem{problem-internal}{Problem}[]
\newenvironment{problem}{
  \medskip
  \begin{problem-internal}
}{
\end{problem-internal}
}

% Solution environment
\newenvironment{solution}{
  \begin{proof}[Solution]
    \vspace{-8px}
    \setlength{\parskip}{4px}
    \setlength{\parindent}{0px}
}{
\end{proof}
}

\begin{document}

  \maketitle

  % Problem 1
  \begin{problem}
    How many positive integers less than or equal to 1000 are divisible by none of 3, 8, and 25?
  \end{problem}

  \begin{solution}
    We can apply the principle of Inclusion-Exclusion to positive integers less than or equal to 1000.
    First note that the number of positive integers less than or equal to 1000 divisible by \(k\) is \(\lfloor{\frac{1000}{k}}\rfloor\). In particular, this counts the number of multiples of \(k\) less than or equal to 1000.
    Then we have
    \begin{displaymath}
      1000 - \lfloor{\frac{1000}{3}}\rfloor - \lfloor{\frac{1000}{8}}\rfloor - \lfloor{\frac{1000}{25}}\rfloor + \lfloor{\frac{1000}{3 \cdot 8}}\rfloor + \lfloor{\frac{1000}{3 \cdot 25}}\rfloor + \lfloor{\frac{1000}{8 \cdot 25}}\rfloor - \lfloor{\frac{1000}{3 \cdot 8 \cdot 25}}\rfloor = 560
    \end{displaymath}
    Thus, there are 560 positive integers less than or equal to 1000 that are divisible by none of 3, 8, and 25.
  \end{solution}

  % Problem 2
  \begin{problem}
    Find the coefficient on \(x^{10}\) in the generating function \(\left(1 + x\right)^{12}\).
  \end{problem}

  \begin{solution}
    We can find the coefficient of \(x^{10}\) by using the binomial theorem, which states that the coefficient of \(x^{10}\) is \({12 \choose 10} = 66\).
  \end{solution}

  % Problem 3
  \begin{problem}
    Solve the recurrence equation \(r_{n+2} = r_{n+1} + 2r_{n}\) if \(r_{0} = 1, r_{2} = 3\).
  \end{problem}

  \begin{solution}
    We can rewrite the recurrence as a polynomial in the advancement operator.
    \begin{align*}
      (A^{2} - A - 2) f(n) = 0
    \end{align*}
    The polynomial in \(A\) factors into \((A - 2)(A + 1)\).
    The corresponding bases for the solution space are \(2^{n}\) and \((-1)^{n}\).
    Thus, the general solution to the recurrence relation is \(r_{n} = c_{0}2^{n} + c_{1}(-1)^n\).

    We're given two conditions. Using the first, we have \(r_{0} = c_{0}2^{0} + c_{1}(-1)^{0}\).
    That is, \(c_{0} + c_{1} = 1\). Similarly, the second condition yields \(8c_{0} + c_{1} = 3\).
    Using the first equation, we solve for \(c_{1} = 1 - c_{0}\), which we can substitute into the second equation.
    This shows that \(7c_{0} = 2 \) which implies that \(c_{0} = \frac{2}{7}\).
    Finally, we can substitute this back into one of the previous equations to solve for \(c_{1} = \frac{5}{7}\). Thus our recurrence is
    \begin{align*}
      r_{n} = \frac{2}{7} \cdot 2^{n} + \frac{5}{7} \cdot (-1)^{n}
    \end{align*}
  \end{solution}

\end{document}
