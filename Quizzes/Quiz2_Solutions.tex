\documentclass[12pt]{article}

% Opening
\title{Applied Combinatorics Quiz 2}
\author{Akash Narayanan}
\usepackage{amsmath, amsfonts, amssymb, amsthm, enumitem, tikz}
\usepackage{caption, subcaption, float, marginnote}

\reversemarginpar

% Putting the margin at the beginnning of the item forces it to print on the same line as the numbering :/

\begin{document}

  \maketitle

  True-False:

  \begin{enumerate}
    \item \marginnote{T} There is a graph \(G\) with \(\omega \left( G \right) = 2\) and \(\chi \left( G \right) = 100\).
    \item \marginnote{T} There is a graph \(G\) with \(\omega \left( G \right) = 3\) and \(\chi \left( G \right) = 100\).
    \item \marginnote{F} There is a planar graph \(G\) with \(\omega \left( G \right) = 2\) and \(\chi \left( G \right) = 100\).
    \item \marginnote{F} If \(\chi \left( G \right) = 3\), then \(G\) is perfect.
    \item \marginnote{T} There is a perfect graph with 240 vertices and 1024 edges.
    \item \marginnote{F} There is a poset with 4215 points having width 79 and height 39.
    \item \marginnote{T} There is a poset with 4215 points having width 97 and height 93.
    \item \marginnote{F} When \(n \geq 3\), the shift graph \(S_{n}\) contains a triangle.
    \item \marginnote{F} When \(n \geq 2\), the shift graph \(S_{n}\) has \(C \left( n, 3 \right)\) vertices.
    \item \marginnote{F} When \(n \geq 2\), the shift graph \(S_{n}\) has \(C \left( n, 2 \right)\) edges.
  \end{enumerate}
\end{document}
