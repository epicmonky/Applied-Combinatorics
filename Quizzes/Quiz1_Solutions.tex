\documentclass[12pt]{article}

% Opening
\title{Applied Combinatorics Quiz 1}
\author{Akash Narayanan}
\usepackage{amsmath, amsfonts, amssymb, amsthm, enumitem, tikz}
\usepackage{caption, subcaption, float, marginnote}

\reversemarginpar

% Putting the margin at the beginnning of the item forces it to print on the same line as the numbering :/

\begin{document}

 \maketitle

 True-False:

 \begin{enumerate}
   \item \marginnote{F} \(P(10, 4) = 720\)
   \item \marginnote{F} \(C(10, 4) = 120\)
   \item \marginnote{F} Any connected graph with an even number of edges has an Euler circuit.
   \item \marginnote{T} There is a connected graph with 500 vertices and 5000 edges which does not have a Hamiltonian cycle.
   \item \marginnote{F} The number of lattice paths from (0, 0) to (12, 12) which pass through (6, 8) is \(C(12, 6)C(12, 8)\)
   \item \marginnote{F} If \(G\) is a graph and \(\chi(G) = 3\), then \(\omega(G) = 3\)
   \item \marginnote{T} If \(G\) is a graph on 20 vertices and every vertex has at least 12 neighbours, then \(G\) has a Hamiltonian cycle.
   \item \marginnote{F} The number of lattice paths from (0, 0) to (12, 12) which do not go above the diagonal is the Catalan number \(\frac{C(12, 6)}{7}\)
   \item \marginnote{T} \(\log{}n = O(\sqrt{n})\)
   \item \marginnote{T} \(\log{}n = o(\sqrt{n})\)
 \end{enumerate}

\end{document}
