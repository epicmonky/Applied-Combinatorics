\documentclass[12pt]{article}

% Opening
\title{Applied Combinatorics Homework 1}
\author{Akash Narayanan}
\usepackage{amsmath, amsfonts, amssymb, amsthm, enumitem}

% Problem environment
\theoremstyle{definition}
\newtheorem{problem-internal}{Problem}[]
\newenvironment{problem}{
  \medskip
  \begin{problem-internal}
}{
\end{problem-internal}
}

% Solution environment
\newenvironment{solution}{
  \begin{proof}[Solution]
    \vspace{-8px}
    \setlength{\parskip}{4px}
    \setlength{\parindent}{0px}
}{
\end{proof}
}


\begin{document}

  \maketitle

  % Problem 1
  \begin{problem}
    How many integer solutions are there to the inequality
    \[ x_1 + x_2 + x_3 + x_4 + x_5 \leq 782 \]
    provided that $x_{1}, x_{2} > 0$, $x_{3} \geq 0$, and $x_{4}, x_{5} \geq 10$?
  \end{problem}

  \begin{solution}
    Consider the inequality as a partitioning of 782 objects. We start by introducing a sixth variable to hold the objects not partitioned among the first five variables, say $x_{6}$. Since the inequality in the stated problem is not strict, $x_{6}$ is not necessarily greater than zero. That is, $x_{6} \geq 0$.

    The variables $x_{1}$ and $x_{2}$ behave as we'd like since they are strictly positive. For $x_{3}$ and $x_{6}$, we add two "artificial" objects that get removed from the variables after the partition if they are non-zero. In consideration of $x_{4}$ and $x_{5}$, we set aside 9 objects for each variable. By operating under the fact that they will both have at least one object after the partitioning, we ensure they will have at least ten objects total.

    Thus, we have 782 + 2 - 18 objects partitioned among 6 variables, yieleding a total of ${755 \choose 5}$ possible solutions.
  \end{solution}

  % Problem 2
  \begin{problem}
    Give a combinatorial argument to prove the identity
    \[ k {n \choose k} = n {n-1 \choose k-1}. \]
  \end{problem}

  \begin{solution}
    Consider a group of $n$ people from which a team of $k$ people are chosen, out of which one person will become captain. The left side of the given equation represents choosing the team of $k$ people and then choosing 1 person out of the $k$ people to be captain. The right side of the equation represents choosing the captain out of the $n$ people and then choosing $k-1$ team members out of the remaining $n-1$ people. \qedhere
  \end{solution}

  % Problem 3
  \begin{problem}
    How many lattice paths from $(0, 0)$ to $(17, 12)$ are there that pass through $(7, 6)$ and $(12, 9)$?
  \end{problem}

  \begin{solution}
    This is equivalent to asking how many paths there are from each point to the following point and finding the product of those counts. In general, between points $(m, n)$ and $(p, q)$ there are $C((p-m)+(q-n), p-m)$ lattice pathsl Thus, we find there are
    \[{(7 - 0) + (6 - 0) \choose (7 - 0)} \cdot {(12 - 7) + (9 - 6) \choose (12 - 7)} \cdot {(17 - 12) + (12 - 9) \choose (17 - 12)} \]
    \[
    \begin{split}
      & = {13 \choose 7} \cdot {8 \choose 5} \cdot {8 \choose 5} \\[2ex]
      & = 5381376 \text{ different paths.} \qedhere
    \end{split}
    \]
  \end{solution}

  % Problem 4
  \begin{problem}
    Let $S$ be the set of strings on the alphabet $\{0, 1, 2, 3\}$ that do not contain 12 or 20 as substring. Give a recursion for the number $h(n)$ of strings in $S$ of length $n$.
  \end{problem}

  \begin{solution}
    We solve this by counting the total number of strings of length \(n\) based on the number of good strings of length \(n-1\) and subtracting the number of bad strings introduced. There are \(4h(n-1)\) possible strings of length \(n\) where the string not including the last digit is good. These are formed by appending one of the four elements of the alphabet to a good string of length \(n-1\).

    Consider the set of strings of length \(n\) ending in 2. For a string in this set to be bad, the preceding digit must be 1. That is, the string preceding the 1 should be good so there are precisely \(h(n-2)\) of these strings.

    Similarly, we consider the set of strings of length \(n\) ending in 0. For one of these strings to be bad, the preceding string must be good and end in 2. There are \(h(n-2) + h(n-3)\) of these strings. This can be interpreted as considering the number of good strings of length \(n-2\) not ending with a 1, as appending a 2 to a string ending with 1 would make it bad before appending the 0.

    Based on manual computation, we find \(h(1) = 4, h(2) = 14, h(3) = 49\). Using these relations, our recursive formula is
    \[h(n) = 4h(n-1) \, - \, 2h(n-2) \, + \, h(n-3) \]
    This yields \(h(4) = 172\) which a manual computation shows is correct.
  \end{solution}

  % Problem 5
  \begin{problem}
    For each formula, give both a proof using the Principle of Mathematical Induction and a combinatorial proof. One of the two will be easier while the other will be more challenging.
    \begin{enumerate}[label={(\alph*)}]
      \item $1^{2} + 2^{2} + 3^{2} + \cdots + n^{2} = \frac{n(n+1)(2n+1)}{6}$
      \item ${n \choose 0} 2^{0} + {n \choose 1} 2^{1} + {n \choose 2} 2^{2} + \cdots + {n \choose n} 2^{n} = 3^{n}$
    \end{enumerate}
  \end{problem}

  \begin{solution}
    We show two proofs of each statement.
    \begin{enumerate}[label={(\alph*)}]
      \item Proofs of statement (a)
      \begin{itemize}
        \item An inductive proof of the first statement begins with a consideration of the base case. Indeed, for \(n=1\), the left side is 1 while the right side is \(\frac{1(2)(3)}{6} = 1\). Now assume the statement holds when \(n = k\) for some \(k \geq 1\). Then we have
        \[
          1^2 + 2^2 + 3^2 + \cdots + k^2 + (k+1)^2 = \frac{k(k+1)(2k+1)}{6} + (k+1)^2
        \]
        using the inductive hypothesis. Upon expanding this we get
        \[
        \begin{split}
          \frac{2k^3 + 3k^2 + k}{6} + k^2 + 2k + 1 & = \frac{2k^3 + 9k^2 + 13k + 6}{6} \\
          & = \frac{(k+1)(k+2)(2k+3)}{6}
        \end{split}
        \]
        Thus, the statement holds when \(n=k+1\) so the statement is true for all \(n \geq 1\) by the Principle of Induction.

        \item A combinatorial proof of the first statement begins by considering a set of k objects. Choose any 2 of these objects \textit{with} replacement. There are precisely \(k^2\) ways to do this. Another way to count this is to split it into cases where the two objects drawn are the same and when they are different. There are \({k \choose 1}\) possibilities where the objects are the same and \(2 {k \choose 2}\) possibilities where the objects are different (it is multiplied by 2 in order to include the opposite order of drawing the objects). Thus we have
        \[ k^2 = {k \choose 1} + 2 {k \choose 2} \]
        Now consider this expression summed over the first \(n\) integers.
        \[ \sum_{k=1}^{n} k^2 = \sum_{k=1}^{n} {k \choose 1} + 2 \sum_{k=1}^{n} {k \choose 2} \]
        We use the combinatorial identity that \(\sum_{i=k}^{n} {i \choose k} = {n+1 \choose k+1}\), which (as shown in the textbook) can be interpreted as the number of bit strings of length \(n+1\) with \(k+1\) 1's, where the last 1 occurs at the \(k\)-th position. This identity reduces the equation to
        \[ \sum_{k=1}^{n} k^2 = {n+1 \choose 2} + 2 {n+1 \choose 3} \]
        noting that in the rightmost sum, \({1 \choose 2} = 0\). Upon algebraic expansion of the terms on the right side, we find that
        \[
        \begin{split}
          \sum_{k=1}^{n} k^2 & = \frac{(n+1)(n)}{2!} + 2 \frac{(n+1)(n)(n-1)}{3!} \\
          & = \frac{n^2 + n}{2} + \frac{n^3 - n}{3} \\
          & = \frac{2n^3 + 3n^2 + n}{6} \\
          & = \frac{n(n+1)(2n+1)}{6}
        \end{split}
        \]
        Thus, we are done. It was primarily a matter of reducing a combinatorial expression to the more recognizable formula.

      \end{itemize}
      \item Proofs of statement (b)
      \begin{itemize}
        \item An inductive proof of the second statement begins by showing the statement holds for the base case \(n=0\). The left side evaluates to \({0 \choose 0} 2^0\) which equals 1. The right side becomes \(3^0\) which is also 1. Thus the statement holds for a base case. Now suppose the statement holds for \(n = k\) where \(k \geq 0\). Then consider the case where \(n = k+1\). We have
        \[
          \sum_{i=0}^{k+1} {k+1 \choose i} 2^i = \sum_{i=0}^{k} {k+1 \choose i} 2^i + 2^{k+1}
        \]
        We first extract the index \(i=0\) from the summation. This is done to avoid having terms \({n \choose k}\) with \(k\) being negative later in the proof. The expression becomes
        \begin{align*}
          \sum_{i=1}^{k+1} {k+1 \choose i} 2^i + 2^0 + 2^{k+1}
        \end{align*}
        Now we use the combinatorial identity \({n+1 \choose k} = {n \choose k} + {n \choose k-1}\) to transform the sum into
        \[
          \sum_{i=1}^{k} \left[{k \choose i} + {k \choose i-1} \right]2^i + 2^0 + 2^{k+1}
        \]
        We separate the sum into two parts by linearity of addition to obtain the expression
        \[
          \sum_{i=1}^{k} {k \choose i} 2^i + {k \choose 0} 2^0 + \sum_{i=1}^{k} {k \choose i-1} 2^i + 2^{k+1}
        \]
        We combine the first sum with the second term to get
        \[
          \sum_{i=0}^{k} {k \choose i} 2^i + \sum_{i=1}^{k} {k \choose i-1} 2^i + 2^{k+1}
        \]
        We apply the inductive hypothesis to the sum on the left and see that this equals
        \[
          3^k + \sum_{i=1}^{k} {k \choose i-1} 2^i + 2^{k+1}
        \]
        We reindex the summation by letting \(j = i-1\) to get the equivalent expression
        \[
          3^k + \sum_{j=0}^{k-1} {k \choose j} 2^{j+1} + 2^{k+1}
        \]
        Finally, we factor a 2 out of the sum and the term on the right to observe that
        \begin{align*}
          \sum_{i=0}^{k+1} {k+1 \choose i}2^{i} &= 3^k + 2 \left( \sum_{j=0}^{k-1} {k \choose j} 2^j + {k \choose k} 2^k \right) \\
          &= 3^k + 2 \left(\sum_{j=0}^{k} 2^j \right) \\
          &= 3^k + 2 \cdot 3^k &\text{(inductive hypothesis)} \\
          &= 3 \cdot 3^k \\
          &= 3^{k+1}
        \end{align*}
        Thus, we have shown that if the statement holds for \(n=k\) then it holds for \(n=k+1\) so by the Principle of Induction the statement is true for all \(n \geq 0\).

        \item A combinatorial proof begins by considering a ternary string of length \(n\). Certainly there are \(3^n\) of these strings. The left side of the equation can be interpreted as partitioning the ternary strings based on the number of 2s. Specifically, \({n \choose k}\) represents a string of length \(n\) where \(k\) positions do not contain a 2. Then there are \(2^k\) ways to fill in these \(k\) positions, where each one contains either a 0 or a 1. Thus, this counts the total number of ternary strings of length \(n\). \qedhere
      \end{itemize}
    \end{enumerate}
  \end{solution}

  % Problem 6
  \begin{problem}
    Show that for all integers $n \geq 4$, $2^{n} < n!$.
  \end{problem}
  \begin{solution}
    We prove the statement by induction. First, we show the statement holds for $n=4$. Indeed,
    \[ 2^{4} = 16 < 24 = 4! \]
    Now assume the statement holds when $n=k$ for some integer $k \geq 4$. That is, $2^{k} < k!$. Then it follows that
    \[2^{k + 1} = 2 \cdot 2^{k} < 2 \cdot k! < (k+1) \cdot k! = (k+1)! \]
    where the second step uses the inductive hypothesis and the third step uses the fact that $(k + 1) > 2$. Thus, the statement holds true for $n = k + 1$ and by the Principle of Mathematical Induction it holds for all $n \geq 4$.
  \end{solution}

  % Problem 7
  \begin{problem}
    Give a proof by induction of the Binomial Theorem. How do you think it compares to the combinatorial argument given in Chapter 2?
  \end{problem}

  \begin{solution}
    The binomial theorem states that
    \[(x+y)^{n} = \sum_{k=0}^{n} {n \choose k} x^{k}y^{n-k}\]
    We begin by showing the statement holds for the base case of \(n=0\). Indeed, the left and right sides are both equal to 1. Now we assume the statement holds for \(n=k\). Then we have that
    \[
    \begin{split}
      (x+y)^{k+1} & = (x+y) (x+y)^{k} \\
      & = (x+y) \sum_{i=0}^{k} {k \choose i} x^{i}y^{k-i} \\
      & = \sum_{i=0}^{k} {k \choose i} x^{i+1}y^{k-i} + \sum_{i=0}^{k} {k \choose i} x^{i}y^{k-i+1} \\
    \end{split}
    \]
    Reindexing the left sum by letting \(r = i + 1\) we obtain the expression
    \[
      \sum_{i=1}^{k+1} {k \choose i-1} x^{i}y^{k+1-i} + \sum_{i=0}^{k} {k \choose i} x^{i}y^{k+1-i}
    \]
    We extract the uppermost term from the left sum and the lowermost term from the right sum and find that
    \[
      {k \choose k} x^{k+1}y^{0} + \sum_{r=1}^{k} {k \choose r} x^{r}y^{k+1-r} + \sum_{i=1}^{k} {k \choose i} x^{k}y^{k+1-i} + {k \choose 0} x^{0}y^{k+1}
    \]
    Combining the two sums (since they have the same indices) and simplifying the terms on the left and right we get
    \[
      x^{k+1} + \sum_{i=1}^{k} \left( {k \choose i-1} + {k \choose i} \right) x^{i}y^{k+1-i} + y^{k+1}
    \]
    We use the combinatorial identity that \({n \choose k} + {n \choose k-1} = {n+1 \choose k}\) to finally obtain
    \[
    \begin{split}
      (x+y)^{k+1} & = x^{k+1} + \sum_{i=1}^{k} {k+1 \choose i} x^{i}y^{k+1-i} + y^{k+1} \\
      & = \sum_{i=0}^{k+1} {k+1 \choose i} x^{i}y^{k+1-i}
    \end{split}
    \]
    where the final equality can be seen by observing the left and right terms as the lowermost and uppermost terms of the summation. Thus, the inductive hypothesis holds for \(n=k+1\) so by the Principle of Induction, the Binomial Theorem holds for all \(n \geq 0\). \qedhere


    The inductive proof of this theorem feels very contrived compared to the combinatorial one, and it reveals little about the nature of why it is true. I certainly would not think of the binomial theorem if I had to do so through induction, whereas through the combinatorial proof it seems like a property that arises almost naturally.
  \end{solution}

  % Problem 8
  \begin{problem}
    Suppose that $x \in \mathbb{R}$ and $x > -1$. Prove that for all integers $n \geq 0, (1+x)^{n} \geq 1 + nx$.
  \end{problem}

  \begin{solution}
    We prove the statement by induction. First, we show that the statement holds for the base case of $n=0$. We see that for all $x > -1$:
    \[ (1+x)^{0} = 1 = 1 + 0 \cdot x \]
    Thus, the statement holds for $n=0$.

    Note that the value $kx^{2} \geq 0$ for all non-negative integers $k$ and all real numbers $x$.

    Now assume the statement holds when $n=k$ for some positive integer $k$. It follows that
    \begin{align*}
          (\forall x \in \mathbb{R},\; x > -1): \quad (1+x)^{k+1} & = (1+x) (1+x)^{k} &\text{(inductive hypothesis)}\\
          & \geq (1+x)(1+kx) \\
          & = 1 + (k+1)x + kx^{2} &\text{\(kx^2 \geq 0\)}\\
          & \geq 1 + (k+1)x
    \end{align*}
    By the Principle of Mathematical Induction, the statement holds for all $n \geq 0$.
  \end{solution}

\end{document}
